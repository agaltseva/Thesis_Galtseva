\documentclass[12pt, fleqn, unicode]{article}
\usepackage{Diplo}
\usepackage{tikz}
\usepackage{mathtools}
\usetikzlibrary{shapes,arrows,shadows}
% \usepackage{subcaption}
% \usepackage{subfig}
\usepackage{subcaption}


% \usepackage{graphicx}
%\usepackage{caption}
%\usepackage{subcaption}
%\usepackage[cp1251]{inputenc}
%\usepackage{amssymb,amsmath}
%\usepackage[russian]{babel}
%\usepackage{array}
%\usepackage[ruled,section]{algorithm}
%\usepackage[noend]{algorithmic}
%\usepackage[all]{xy}
%\usepackage{graphicx}
%\usepackage{indentfirst}
% \usepackage{bigstrut}
% \usepackage{float}
% \DeclareMathOperator*{\argmax}{arg\,max}

%\clubpenalty=10000
%\widowpenalty=10000
%% для больших плавающих иллюстраций
%\renewcommand\topfraction{1.0}
%\renewcommand\textfraction{0.0}
%\renewcommand\floatpagefraction{0.01} % float-страниц быть вообще не должно - это чтобы их лучше видеть ;)
% Воронцов
% \newcommand\brop[1]{#1\discretionary{}{\hbox{$\mathsurround=0pt #1$}}{}}
\newcommand{\tsum}{\mathop{\textstyle\sum}\limits}
\renewcommand{\geq}{\geqslant}
\renewcommand{\leq}{\leqslant}
\newcommand{\scal}[2]{\left\langle #1,#2 \right\rangle}
\def\XYtext(#1,#2)#3{\rlap{\kern#1\lower-#2\hbox{#3}}}

\newcommand{\executeiffilenewer}[3]{%
 \ifnum\pdfstrcmp{\pdffilemoddate{#1}}%
 {\pdffilemoddate{#2}}>0%
 {\immediate\write18{#3}}\fi%
}
\newcommand{\includesvg}[1]{%
 \executeiffilenewer{#1.svg}{#1.pdf}%
 {inkscape -z -D --file=#1.svg %
 --export-pdf=#1.pdf --export-latex}%
 \input{#1.pdf_tex}%
}
\usepackage{pb-diagram}
\usepackage{bm}

\graphicspath{ {./fig/} }

%%%%%%%%%%%%%%%%%%%%%%%%%%%%%%%%%%%%%%%%%%%%%%%%%%%%%%%%%%%%%%%%%%%%%%%%%%%%%%%
% Оформление алгоритмов в пакетах algorithm, algorithmic
%%%%%%%%%%%%%%%%%%%%%%%%%%%%%%%%%%%%%%%%%%%%%%%%%%%%%%%%%%%%%%%%%%%%%%%%%%%%%%%
% % переопределения (русификация) управляющих конструкций:
% \newcommand{\algKeyword}[1]{{\bf #1}}
\floatname{algorithm}{Алгоритм}
\renewcommand{\algorithmicrequire}{\rule{0pt}{2.5ex}{{\bf Вход:}}}
\renewcommand{\algorithmicensure}{{\bf Выход:}}
\renewcommand{\algorithmicend}{{\bf конец}}
\renewcommand{\algorithmicif}{{\bf если}}
\renewcommand{\algorithmicthen}{{\bf то}}
\renewcommand{\algorithmicelse}{{\bf иначе}}
\renewcommand{\algorithmicelsif}{\algorithmicelse\ \algorithmicif}
\renewcommand{\algorithmicendif}{\algorithmicend\ \algorithmicif}
\renewcommand{\algorithmicfor}{{\bf для}}
\renewcommand{\algorithmicforall}{{\bf для всех}}
\renewcommand{\algorithmicdo}{}
\renewcommand{\algorithmicendfor}{\algorithmicend\ \algorithmicfor}
\renewcommand{\algorithmicwhile}{{\bf пока}}
\renewcommand{\algorithmicendwhile}{\algorithmicend\ \algorithmicwhile}
\renewcommand{\algorithmicloop}{{\bf цикл}}
\renewcommand{\algorithmicendloop}{\algorithmicend\ \algorithmicloop}
\renewcommand{\algorithmicrepeat}{{\bf повторять}}
\renewcommand{\algorithmicuntil}{{\bf пока}}
%\renewcommand{\algorithmiccomment}[1]{{\footnotesize // #1}}
\renewcommand{\algorithmiccomment}[1]{{\quad\sl // #1}}

\newcommand{\bz}{\mathbf{z}}
\newcommand{\bx}{\mathbf{x}}
\newcommand{\by}{\mathbf{y}}
\newcommand{\bw}{\mathbf{w}}
\newcommand{\bY}{\mathbf{Y}}
\newcommand{\bX}{\mathbf{X}}
\newcommand{\ba}{\mathbf{a}}
\newcommand{\bu}{\mathbf{u}}
\newcommand{\bt}{\mathbf{t}}
\newcommand{\bp}{\mathbf{p}}
\newcommand{\bq}{\mathbf{q}}
\newcommand{\br}{\mathbf{r}}
\newcommand{\bg}{\mathbf{g}}
\newcommand{\bh}{\mathbf{h}}
\newcommand{\bb}{\mathbf{b}}
\newcommand{\bv}{\mathbf{v}}
\newcommand{\be}{\mathbf{e}}
\newcommand{\bc}{\mathbf{c}}
\newcommand{\bs}{\mathbf{s}}
\newcommand{\bP}{\mathbf{P}}
\newcommand{\bT}{\mathbf{T}}
\newcommand{\bQ}{\mathbf{Q}}
\newcommand{\bE}{\mathbf{E}}
\newcommand{\bF}{\mathbf{F}}
\newcommand{\bU}{\mathbf{U}}
\newcommand{\bI}{\mathbf{I}}
\newcommand{\bB}{\mathbf{B}}
\newcommand{\bW}{\mathbf{W}}
\newcommand{\bD}{\mathbf{D}}
\newcommand{\bH}{\mathbf{H}}
\newcommand{\bG}{\mathbf{G}}
\newcommand{\bS}{\mathbf{S}}
\newcommand{\bZ}{\mathbf{Z}}
\newcommand{\bJ}{\mathbf{J}}
\newcommand{\bM}{\mathbf{M}}
\newcommand{\btheta}{\boldsymbol{\theta}}
\newcommand{\bmu}{\boldsymbol{\mu}}
\newcommand{\blambda}{\boldsymbol{\lambda}}
\newcommand{\bPsi}{\boldsymbol{\Psi}}
\newcommand{\bchi}{\boldsymbol{\chi}}
\newcommand{\bsigma}{\boldsymbol{\sigma}}
\newcommand{\bTheta}{\boldsymbol{\Theta}}
\newcommand{\bphi}{\boldsymbol{\phi}}
\newcommand{\bdelta}{\boldsymbol{\delta}}

\newcommand{\T}{^{\text{\tiny\sffamily\upshape\mdseries T}}}


\newcommand{\R}{\mathbb{R}}
\newcommand{\Q}{\mathbb{Q}}
\renewcommand{\C}{\mathbb{C}}
\newcommand{\N}{\mathbb{N}}
\newcommand{\Z}{\mathbb{Z}}
\newcommand{\E}{\mathbb{E}}
\newcommand{\var}{\mathrm{Var}\;}
\newcommand{\diam}{\mathrm{diam}\;}
\newcommand{\conv}{\mathrm{conv}\;}
\newcommand{\cl}{\mathrm{cl}\;}
\newcommand{\dist}{\mathbf{dist}}
\newcommand{\dom}{\mathbf{dom}\;}
\renewcommand{\sign}{\mathbf{sign}\;}
\renewcommand{\T}{\intercal}
\renewcommand{\eps}{\varepsilon}


% Fitting braces
\newcommand{\brs}[1]{\left(#1\right)}
\newcommand{\sbrs}[1]{\left[#1\right]}
\newcommand{\fbrs}[1]{\left\{#1\right\}}
\newcommand{\rbrs}[1]{\left\langle #1 \right\rangle}

\makeatletter
\newenvironment{sqcases}{%
  \matrix@check\sqcases\env@sqcases
}{%
  \endarray\right.%
}
\def\env@sqcases{%
  \let\@ifnextchar\new@ifnextchar
  \left\lbrack
  \def\arraystretch{1.2}%
  \array{@{}l@{\quad}l@{}}%
}
\makeatother

\DeclarePairedDelimiter\ceil{\lceil}{\rceil}
\DeclarePairedDelimiter\floor{\lfloor}{\rfloor}
\DeclarePairedDelimiter\abs{\lvert}{\rvert}%
\DeclarePairedDelimiter\norm{\lVert}{\rVert}%

% Swap the definition of \abs* and \norm*, so that \abs
% and \norm resizes the size of the brackets, and the
% starred version does not.
\makeatletter
\let\oldabs\abs
\def\abs{\@ifstar{\oldabs}{\oldabs*}}
%
\let\oldnorm\norm
\def\norm{\@ifstar{\oldnorm}{\oldnorm*}}
\makeatother


\begin{document}

{
\renewcommand{\baselinestretch}{1}
\thispagestyle{empty}
\begin{center}
    \sc
        Министерство образования и науки Российской Федерации\\
        Московский физико-технический институт
        {\rm(государственный университет)}\\
        Факультет управления и прикладной математики\\
        Кафедра <<Интеллектуальные системы>>\\
        при Вычислительном центре им. А. А. Дородницына РАН\\[35mm]
    \rm\large
        Гальцева Александра Ивановна\\[10mm]
    \bf\Large
    Модели инвариатных и композитных движений в задачах анализа физической активности\\[10mm]
    \rm\normalsize
        030401 — Прикладные математика и физика\\[10mm]
    \sc
        Выпускная квалификационная работа магистра\\[30mm]
\end{center}
\hfill\parbox{80mm}{
    \begin{flushleft}
    \bf
        Научный руководитель:\\
    \rm
        д.ф.-м.н. Стрижов Вадим Викторович\\[4.9cm]
    \end{flushleft}
}
\begin{center}
    Москва\\
    2020 г.
\end{center}
}

\newpage
\tableofcontents

%%%%%%%%%%%%%%%%%%%%%%%%%%%%%%%%%%%%%%%%%%%%%%%%%%%%%%%%%%%%%%%%%%%%%%%%%%%%%%
\newpage
\begin{abstract}


  \bigskip
    \textbf{Ключевые слова}: \emph{разложение временных рядов, модели инвариантных движений.}
\end{abstract}

%%%%%%%%%%%%%%%%%%%%%%%%%%%%%%%%%%%%%%%%%%%%%%%%%%%%%%%%%%%%%%%%%%%%%%%%%%%%%%
\newpage
\section{Введение}
%%%%%%%%%%%%%%%%%%%%%%%%%%%%%%%%%%%%%%%%%%%%%%%%%%%%%%%%%%%%%%%%%%%%%%%%%%%%%%

За последние годы накоплено много данных, которые описывают физическую активность, акселерометрамми сматрфонов и иными устройтвами
(фитнес-браслеты, медициское оборудование). Данные представлят собой
временные ряды. Решается задача классификации видов деятельности человека по измерениям акселерометра. Работа посвящена исследованию проблемы декомпозиции движения на элементарные посредством исследования пространства признаков ЛАМ.
% [, проверяется гипотеза существования алгоритма декомпозиции суперпозиции элементарных движений.]
Предполагается, что временной ряд предаставляет собой 1) композицию/суперпозицию или 2)  объединение элементарных движений.
 Рассмотрены несколько подходов разложения временных рядов: декопмпозиция временного ряда и разложение суммы\произведения движений. В первом случае производится вычисление обратной функции к функции движения (??), а во втором - разделение пространства параметров ЛАМ на подпространства.   
% Задача декомпозиции движения является подзадачей в процессе классификации движений.
В работе~\cite{Kuznetsov2015} рассматривается решение задачи классификаци физической активности посредством постоения промежуточного признакового пространства. В этой работе оно будет исследовано. Предполагается, что это пространство с метрикой $\mu$. Также в пространстве параметров ЛАМ определен инвариант физических характеристик испытуемого[/действующего лицо/движущегося объекта.(???)  
% Это может быть мнимая часть компонентов ряда (в случае отображения ряда в комплексную оласть) или амплитуда располодения параметров ЛАМ в метрическом пространстве.
Исследование пространства параметров ЛАМ состоит из двух этапов:
\begin{itemize}
    \item На первом этапе исследуется пространство параметров ЛАМ: в каком метрическом пространстве оно находится, введена метрика, понятие инварианта движения в данном пространстве.
    \item На втором этапе производится разложение пространства параметров ЛАМ на подпространства, чтобы показать, что оно есть элементарное движение. Также осуществлена попытка найти алгоритм декомпозиции и разбиения пространтва.
\end{itemize}

В результате проделанной работы получаем классификатор элементарных движений, который дает возможность решать задачи детектирования, ...
Таким образом, алгоритм позволяет найти/определить уникальный шаблон для всех движений - инвариант.



%%%%%%%%%%%%%%%%%%%%%%%%%%%%%%%%%%%%%%%%%%%%%%%%%%%%%%%%%%%%%%%%%%%%%%%%%%%%%%%
\newpage
\section{Обзор литературы}
%%%%%%%%%%%%%%%%%%%%%%%%%%%%%%%%%%%%%%%%%%%%%%%%%%%%%%%%%%%%%%%%%%%%%%%%%%%%%%%
% Переформулировать и добавить еще обзоров 
Алгоритм классификации временных рядов по их признаковому описанию а также
базовые подходы к генерации признаковых описаний в
 задаче определения типа движения рассмотрены в работе~\cite{Kuznetsov2015}.
Альтернативный подход к генерации признаковых описаний, основанный на описании
сегментов оптимальными параметрами аппроксимирующих моделей описан в
работе~\cite{Karasikov2016}.

%%%%%%%%%%%%%%%%%%%%%%%%%%%%%%%%%%%%%%%%%%%%%%%%%%%%%%%%%%%%%%%%%%%%%%%%%%%%%%
\newpage
\section{Постановка задачи декомпозиции временного ряда}
%%%%%%%%%%%%%%%%%%%%%%%%%%%%%%%%%%%%%%%%%%%%%%%%%%%%%%%%%%%%%%%%%%%%%%%%%%%%%%

 Введем определения временного ряда, сегмента временного ряда.
\begin{Def}
Временной ряд - функция, которая определена на множестве временных меток, отображающая их в $m-$ мерное пространство ($m-$количество используемых датчиков)
 $$S: T\rightarrow\mathbb{R},\text{где } T=\{t_0, t_0 + d, t_0 + 2d, ...\}$$
\end{Def}
Разделим ряд на отрезки - сегменты.
\begin{Def}
Сегмент временного ряда:
$$\boldsymbol{x}_i = [S(t_i), S(t_i - d), ..., S(t_i - (n - 1)d))], \boldsymbol{x}_i \in X\equiv\mathbb{R}^n$$
\end{Def}

Пусть пространство сегментов $X$ отображается набором функций $\boldsymbol{h} = [\boldsymbol{h}_1, ..., \boldsymbol{h_k}]$ в пространство признаковых описаний $G$.
 Нужно найти алгоритм $\mathcal{A}$:  1) $\mathcal{A}(G) = \cup_{i} g_i$, или 2) $\mathcal{A}(G) = (\circ, g_i) $. Таким образом, $\mathcal{A}$ определяет биекцию между $X$ и $G$.

% Оптимальные параметры $\boldsymbol{H}(\boldsymbol{x})$ определяются как
% $$\boldsymbol{h}_i(\boldsymbol{x}) = \arg\min \limits_{\boldsymbol{w} \in \mathbb{R}^{n_g}} \ro(g(\boldsymbol{w}, \boldsymbol{x}), \boldsymbol{x})$$
% $\boldsymbol{h}_i$ - модель локальной аппроксимации.

%     Выборка: $\mathcal{D} = \{(\boldsymbol{x}_i, y_i)\}^N_{i = 1},~ y_i\in \{1, ..., K\}$\\
%     $X$ - набор сегментов данных акселерометра,\\
%     $y$ - метки классов движения,

% \begin{Def}
%  Моделью локальной аппроксимации $g$ называется модель, аппроксимирующая временной ряд $x(t)$ на отрезке времени $[t, t - \Delta t]$:
            % $$g: [t, t - \Delta t]\rightarrow \boldsymbol{\hat{x}}$$
% \end{Def}



% Метод


\subsection{Исследование метрического пространства параметров ЛАМ}

\subsection{Виды отображения движения в метрическое пространство}


%%%%%%%%%%%%%%%%%%%%%%%%%%%%%%%%%%%%%%%%%%%%%%%%%%%%%%%%%%%%%%%%%%%%%%%%%%%%%%
\newpage
\section{Исследование свойств метрического пространства}
%%%%%%%%%%%%%%%%%%%%%%%%%%%%%%%%%%%%%%%%%%%%%%%%%%%%%%%%%%%%%%%%%%%%%%%%%%%%%%


\subsection{Рассмотрение метрик}

\subsubsection{Метрика_1}
...
\subsubsection{Метрика_2}


\subsection{Инвариант элементарного движения в метрическом пространстве}


%%%%%%%%%%%%%%%%%%%%%%%%%%%%%%%%%%%%%%%%%%%%%%%%%%%%%%%%%%%%%%%%%%%%%%%%%%%%%%
\newpage
\section{Композитность движения}
%%%%%%%%%%%%%%%%%%%%%%%%%%%%%%%%%%%%%%%%%%%%%%%%%%%%%%%%%%%%%%%%%%%%%%%%%%%%%%



\subsection{Декомпозиция движений}

\subsubsection{Разложение Фурье}

\subsubsection{Разложение Хаара-Олина}

% \subsubsection{Отображение в комплексную плоскость}
\subsection{Разбиение движений}

%%%%%%%%%%%%%%%%%%%%%%%%%%%%%%%%%%%%%%%%%%%%%%%%%%%%%%%%%%%%%%%%%%%%%%%%%%%%%%
\newpage
\section{Сбор и подготовка данных}
%%%%%%%%%%%%%%%%%%%%%%%%%%%%%%%%%%%%%%%%%%%%%%%%%%%%%%%%%%%%%%%%%%%%%%%%%%%%%%

\subsection{Выделение, выравнивание и разметка сегментов}



\newpage
\section{Вычислительный эксперимент}



\newpage
\section{Выводы}

Удалось показать, что элементарные движения соответствуют подпространствам пространства параметров ЛАМ и найдены способы их выделения. Также найден алгоритм декомпозиции пространства параметров движения .
% Тут осуществлен переход в область топологических пространств и исследованы метода возврата 
Проведен численный эксперимент, подверждающий, что выдвинутая гипотеза существования алгоритма декомпозиции временного ряда может быть принята.  


\newpage
\nocite{*}\

\bibliographystyle{unsrt}
\bibliography{papers.bib}


% \subsubsection{Фурье-модель}

% В качестве аппроксимирующей модели сегмента берем обратное дискретное
% преобразование Фурье,
% то есть признаковым описанием сегмента является прямое дискретное преобразование
% Фурье.
% $$
%     w_{2j} = \mathrm{Re} \sum_{k=1}^{n} x_k \exp\brs{-\frac{2\pi i}{n}kj}, j=1\ldots n
% $$
% $$
%     w_{2j + 1} = \mathrm{Im} \sum_{k=1}^{n} x_k \exp\brs{-\frac{2\pi i}{n}kj}, j=1\ldots n
% $$
% $$
% w_{\text{FFT}} = \sbrs{w_1\ldots w_{2n}}\in \R^{2n}
% $$

% В качестве признеакового описания берем $m$ частот из прямого преобразования
% Фурье с максимальной амплитудой, где $m$ — параметр модели.




\end{document}
